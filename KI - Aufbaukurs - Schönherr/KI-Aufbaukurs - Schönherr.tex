\documentclass[a4paper,14pt]{article}
\usepackage[utf8]{inputenc}
\usepackage{amsmath}
\usepackage{amsfonts}
\usepackage{listings}
\usepackage{float}
\usepackage[ngerman]{babel}
\usepackage{enumerate}

\title{KI - Aufbaukurs - Schönherr}
\author{Hans-Christian Heinz\\
hheinz@imn.htwk-leipzig.de\\
CC-BY-NC-SA 4.0}
\date{WS2012/13}
\begin{document}

\maketitle

\tableofcontents


\section{Wissensrepräsentation (ausgewählte Kapitel)}
\begin{itemize}
 \item Prädikatenlogik vorausgesetzt
 \item Frames vorausgesetzt
\end{itemize}

\subsection{Semantische Netze}

\subsubsection{Grundlagen}


Wunsch nach grafischer Veranschaulichung prädikatenlogischer Ausdrücke und natürlichsprachlicher Sätze\\
\\
Idee: Darstellung von 2-stelligen Prädikaten zwischen Objekten durch Graphen\\
\\
\begin{description}
 \item[Knoten:]konkrete oder abstrakte Objekte
 \item[Kanten:]Beziehungen (binäre Relationen)
\end{description}

Problem: 2-Stelligkeit der Relation

Verwendung verschiedener Kantentypen, die bezeichnet werden ($\to$ verschiedene Prädikate)

==> Semantisches Netze

p(a,b) wird repräsentiert durch $a \to^{p} b$

\glqq{}Standardisierte Namen\grqq{} für häufige Kantennamen

\begin{itemize}
 \item is\_a	isa	-	Klasse-Instanz
 \item a\_kind\_of	ako	-	Klasse-Unterklasse
 \item is\_part\_of
 \item is\_connected\_with
 \item $\vdots$
\end{itemize}


Beispiel:\\
\\
\\
\\

Vorteile:
\begin{itemize}
 \item effiziente Repräsentation (Graphen
 \item Rel. zwischen Objekten entsprechen Zugriffspfaden
 \item Wissen organisierbar (z.B. Hierarchien)
\end{itemize}

Nachteile:
\begin{itemize}
  \item Probleme mit Ableitungen
  \item deklarative Semantik erklärbar über Abbildung auf eine äquivalente Logikrepräsentation
  \item keine Standards
  \item Gefahr der Unübersichtlichkeit
\end{itemize}
  
\subsubsection{Nicht-2-stellige Prädikate}

\begin{enumerate}[(a)]
 \item  1-stellige Prädikate\\
	Einführung eines Dummy-Symbols \emph{d}
	p(a) wird repräsentiert durch
	$$a \xrightarrow{p} d $$
	(also p(a,d)\\
	oder (alternativ zum Dummy - d)
	$$a\xrightarrow{isa}p $$
	\glqq Typ p\grqq{} wird jetzt als Objekt betrachtet (Klasse)
\item 	Darstellung eines 3-stelligen Prädikates p(a,b,c)\\
	Relation (Prädikat) p wird aufgespalten
	Einführung
	\begin{itemize}
	\item eines \glqq Ereignisses\grqq{} e
	\item dreier Ersatz-Prädikate
	\end{itemize}
	$$p(a,b,c)\leftrightarrow p_1(e,a)\wedge p_2(e,b) \wedge p_3(e,c) $$
	Preis: Veränderung der Sprache
\item	Am Knoten kann auch ein Netz stehen ($\to$Hierarchien)
\end{enumerate}

\subsubsection{Darstellung allgemeiner prädikatenlogischer Ausdrücke}

\begin{enumerate}[(a)]
 \item Skolemisierung\\
	Bsp. (Nilsson: Principles of AI)\\
	\glqq{}John gab Mary das Buch\grqq{}\\
	$\downarrow$prädikatenlogische Darstellung\\
	geben(john,mary,book)\\
	ge(neue Konst.): Menge von Gebe-Ereignissen\\
	\begin{equation}
	 geben(john,mary,buch) \leftrightarrow \exists X E (X,ge)\wedge geben(X,john) \wedge empfaenger(X,mary) \wedge objekt(X,buch)
	\end{equation}
	
	Skolemisierung(Skolem.-Konst. g für X) ergibt\\
	\begin{equation}
	 E(g,ge)\wedge empfaenger(g,mary)\wedge geber(g,john) \wedge objekt(g,buch)
	\end{equation}

	oder: Einführung von (neuen) Funktionen\\
	\glqq = \grqq{} bedeute die Gleichheit von Termen\\
	Nach Skolemisierung folgt\\
	\begin{equation}
	 E(g,ge) \wedge =(geber(g),john) \wedge =(empfaenger(g),mary) \wedge =(objekt(g),buch)
	\end{equation}
	
\item 	Konjunktion: beide Pfeile im Netz

\item 	Disjunktion: p(a,b) $\vee$ q(c,d)
	\begin{equation}
	 a \xrightarrow{p} b
	\end{equation}
	\begin{equation}
	 c \xrightarrow{q} d
	\end{equation}
	\item Negation: Markierung der Prädikate (wie bei Disjunktion)\\
	Beispiel:
	\begin{equation}
	 \neg(p(a,b)\wedge q(b,c) \cong \neg p(a,b,) \vee \neg p(b,c)
	\end{equation}
	
\item	Implikation\\
	Ausnutzung von $(p(\bullet,\bullet)\to q(\bullet,\bullet))\cong (q(\dots)\vee p(\dots)$

\item	Quantoren\\
	Var. an Knoten kann
	\begin{itemize}
	 \item existenziell
	 \item allquantifiziert
	\end{itemize}
	sein\\
	Bei \glqq$\exists$\grqq{} - vorher Skolemisierung\\
	Bsp.: \glqq{}John gab jedem irgendetwas.\grqq{}\\
	Logikrepräsentation mit 2-stelligen Prädikaten:
	\begin{equation}
	 \forall X \exists Y \exists Z \{ E(Y,ge) \wedge geben(Y,john) \wedge empfaenger(Y,X) \wedge objekt(Y,Z)\}
	\end{equation}
	Skolemisierung (Beseitigung von \glqq$\exists$\grqq{})
	\begin{equation}
	 \forall X \{ E(g(X),ge)\wedge geber(g(X),X) \wedge objekt(g(X),o(X))\}
	\end{equation}
	g(x) ist eine Skolemfunktion
	\begin{equation}
	 john \xrightarrow{geber}g(X)\xrightarrow{empfaenger}X\xrightarrow{E}mensch
	\end{equation}
	%siehe Franz' Mitschrift für komplette grafik 
\end{enumerate}

\subsubsection{Nutzung semantischer Netze}
Günstig, wenn 2-stellige Präd. Normalfall sind\\
$\bullet,\bullet \wedge \bullet$\\
Netzwerk-Sprachen für 
\begin{itemize}
 \item Manipulation in Netzen
 \item Suche nach Knoten, Kanten, Wegen |$\to$Inferenz (Ableiten)
 \item Umgang mit Funktionen
\end{itemize}

\subsection{Constraint-Netze}
Constraint: Relation (bzw. Prädikat) $p(X_1,..X_n)$ \\
einschränkende Bedingung für Variablen oder eine Gleichung (spez. Fall)\\
\\
Constraint-Netz: Graph, bei dem
\begin{itemize}
 \item Knoten: Relation zw. Obekten repräsentieren
 \item Kanten: Gemeinsame Argumente zweier Rel-Symbole repräsentieren
\end{itemize}

\subsection{Analoge WR}
Synonym: Direkte WR\\
keine wirkliche Modellierung\\
Repräsentation in einer Syntax, die der realen Welt \glqq{}analog\grqq{} ist.
Bsp.:
\begin{itemize}
 \item Bilder
 \item Landkarten
 \item Musiknoten
 \item chemische Formel
 \item grafische Darstellung
 \item $\vdots$
\end{itemize}

Syntax nicht einheitlich, dafür zweckspezifisch\\
Semantik intensional\\
\\
Vorteile:\\
Kleine Änderungen in Realität $\leftrightarrow$ kleine Änderungen in der Repräsentation\\
hoher Informationsgehalt\\
Nachteile:\\
Probleme beim autom. Auswerten des Wissens\\
Aktualisierung des Wissens, Folgen\\

\subsection{Skripte}
Strukturierte Beschreibung stereotyper Abläufe (z.B. Auto starten)\\
\\
Repräsentation ovn Abläufen, Bed. dafür, erforderliche mathematische Voraussetzungen\\
Beschreibung typischer Szenen $\rightarrow$ Ähnlichkeit zu frame-artiger Darstellung (ähnlich wie Objektorientierung)\\
\\
Übungsaufgabe: \glqq{}Der Dekan schüttelte zur Begrüßung jedem die Hand.\grqq{} modellieren als semantisches Netz

\section{KI-Suchmethoden}
\subsection{Optimierung und heuristisches Problemlösen}
Problemlösung in KI-Anw.
\begin{itemize}
 \item Algorithmen
 \item allg. Rahmen für Problemlösungen (Problemlöse-Paradigmen
\end{itemize}

KI-Algorithmen sind meist heuristischer Natur\\
oft diskrete Optimierungsaufgaben ($\rightarrow$Graphensuche)\\
diskret:Analysis nicht anwendbar\\
Terminologie
$$G = (K,R) ~~~~~~~~R\subseteq K x K $$
K: Knotenmenge\\
R: Relation, 2-stellig;Kantenmenge\\
O - Menge von Operationen
$$o\in O, o: K\rightarrow K$$

K oft nicht explizit gegeben, im Prinzip jeder Knoten berechenbar\\
o interpretierbar als Zustandstransformation\\
\\
Meist Zustandsfolge Z (implizit) gegeben, Z $\subset$K\\
Zustandsfolge (mit Op.) gesucht mit Endzustand $\in$ Z\\
\\
Problem P: P = [a,K,Z,O]\\
a: Anfangszustand, a $\in$ K
Knotenarten:
\begin{itemize}
 \item ODER-Knoten zur Beschreibung alternativer Vorgehensweisen
 \item UND-Knoten zur Beschreibung der Aufspaltung eines Problems in Teilprobleme, die alle zu lösen sind
\end{itemize}

\subsection{Problemlöse-Paradigmen}
\subsubsection{Erzeuge-und-teste-Prinzip (generate and test method)}
2 Programm- / Komponenten
\begin{itemize}
 \item Generator zur Erzeugung möglicher Lösungen
 \item Tester zum Testen einer Lösung auf Brauchbarkeit
\end{itemize}
Abbruch-Bedingung.:
\begin{itemize}
 \item akzeptable Lösung gefunden, oder
 \item bei mehreren Lösungen auswählen, oder
 \item Unlösbarkeit feststellen
\end{itemize}

%Programmablaufplan, von Sam oder Franz holen

Anwendung bei Identifikationsproblemen\\
Fall liegt vor.\\
Mit welchem bekannten Fall ist er identisch/ähnlich?\\
($\rightarrow$ fallbasiertes Schließen)\\
Anforderungen an den Generator:
\begin{itemize}
 \item Vollständigkeit, d.h. Generierbarkeit \underline{aller} möglichen Lösungskandidaten
 \item Redundanzfreiheit
 \item Informiertheit: Wissen über Enschränkung der Möglichkeiten einbringen
\end{itemize}

\subsection{Problemlöse-Paradigmen}
\subsubsection{Erzeuge-und-teste-Prinzip}
Beispiele
\begin{enumerate}[(a)]
 \item SEND/MORE=MONEY
 \item Safe-Knacken
 \item DENDRAL (Chemie-Expertensystem, seit 50 Jahren im Einsatz)\\
 Problemstellung: chem. Substanz zu identifizieren\\
 Zuerst:Bestimmung der Anzahl der Atome jeder Sorte in einem Molekül\\
 Beschuss der Moleküle mit hochenergetischen Elektronen\\
 $\rightarrow$Zerbrechen der Moleküle in geladene Teilchen verschiedener Größe\\
 Bruchstücke werden durch ein Magnetfeld geschickt\\
 Generierung von Isomeren(Simulation)\\
 Generierung von Massenspektrogrammen zu diesen Isomeren\\
 Vergleich (Mustererkennung) mit realen Spektrogrammen\\
\end{enumerate}

\subsubsection{Differenzmethode (means-ends analysis}
G=(K,R) geg.\\
Idee:\\
k - Problemlöse-Zustand k $\in$K
aktueller Zustand $\xrightarrow{? Folge von Uebergaengen gesucht}$gewünschter Zustand z$\in$Z\\
Erzeugung eines Nachfolge-Zustandes, der \glqq{}dichter\grqq{} am Zielzustand liegt (besseres Zwischenergebnis)\\
Voraussetzung: \glqq{}Entfernung\grqq{} zum Ziel zu definieren\\
Metrik d(k,z) zwischen Zuständen definieren\\
Verfahren
\begin{itemize}
 \item Expandieren eines Knotens\\
 ~~~~~~$\rightarrow$Berechnung der Nachfolger (1 Schritt)
 \item
\end{itemize}

Probleme z.B. bei Irrgarten-Aufgaben: Umwege, Sackgassen\\
Bsp.: Routenplanung

\subsubsection{Problemreduktion (goal reduction)}
Idee: Zerlegung eines Problems in leichter lösbare Teilprobleme\\
$\rightarrow$ Generierung eines Problembaumes\\
Wurzel: geg. Problem\\
$\vdots$\\
Blätter: relativ einfach lösbare Einzelprobleme\\
i.A. UND-ODER-Baum\\
\begin{itemize}
 \item verschiedene Zerlegunsarten (ODER)
 \item für konkrete Zerlegung alle Teilprobleme lösen (UND)
\end{itemize}

Bsp.:
\begin{enumerate}[(a)]
 \item Turm von Hanoi\\
 %Grafik Hanoi
 Turm der Höhe n gegeben\\
 (1) \~ n-1 umsetzen\\
 (2) \~ 1 umsetzen\\
 (3) \~ n-1 umsetzen\\
 \item Konfiguration von Tagebau Bandanlagen (1986-1990)\\
 UND-ODER\\
 %Grafig Tagebau-Bandanlage
\end{enumerate}

\subsection{Suchalgorithmen ohne Kostenfunktion}
Gegenstand: diskrete Suchprobleme in Graphen\\
Problem: 
\begin{itemize}
 \item Finden eines ausgezeichneten Knotens (Schatzsuche)
 \item Finden eines bestimmten Weges
\end{itemize}

Vorgehen:
\begin{itemize}
 \item Erzeugen eines Suchbaumes der nach bestimmten Prinzip ($\rightarrow$verschiene Algorithmen) abgesucht wird
\end{itemize}

Graphensuche $\rightarrow$ Suche im Baum\\
\\
%Grafik Graph
\\
Weg von s nach z gesucht\\
\\
%Grafik Baum

\subsubsection{Tiefe-zuerst-Suche}
%Grafik Tiefe zuerst Suche

Vor. für Finden von Z
\begin{itemize}
 \item endlicher Graph
 \item Es existiert mind. 1 Zielknoten
 \item hinreichende Vernetzung
\end{itemize}

\subsubsection{Breite-zuerst-Suche}
wie Tiefensuche\\
einzige Änderung: nicht expandierte Nachbarn \underline{hinten} an OPEN anfügen\\

\subsubsection{Nichtdeterministische Suche}
Kompromiss zwischen Tiefen- und Breitensuche\\
Programmablaufplan wie in 2.3.1 aber Änderung des 1. Befehls in Schleife\\
Auswahl eines zufälligen Knotens aus OPEN

\subsection{Suchalgorithmen mit Kostenfunktion}
\subsubsection{Bergsteigen (hill climbing)}
G=(K,R) geg.\\
B - Bewertungsfunktion B: K$\rightarrow$R\\
feste Bewertung für Zielfunktion\\
s$\in$K sei gegeben (Startknoten = Anfangszustand)\\
z$\in$K gesucht, so dass für alle k$\in$K gilt B(z)$\geq$B(k)\\
\\
\\

%KW 44 fehlt

%KW 45


Weitere Bsp.:
\begin{enumerate}[(a)]
 \item Breitensuche in einem Baum\\
	Wähle $\forall(k,n) (k,n)\in\mathbb{R}\rightarrow e(k,n)=1$\\
	$\forall k h(k)=0 $
	$\rightarrow$ g(k) ist der Rang von k
 \item Tiefensuche in einem Baum\\
	Wähle f(s)=0 und $\forall(k,n), (k,n)\in\mathbb{R}\rightarrow f(n)=f(k)-1$\\
	Auswahl des Knotens mit negativster Bewertung
 \item Backtracking mit beschränkter Tiefe\\
       Tiefensuche wie bei b), aber Schranke für f vorgegeben\\
       Entfernung eines Knotens von OPEN, wenn Schranke unterschritten wird
  \item Branch-and-Bound-Algorithmus ergibt sich als Spezialfall $\forall k h(k)=0$
\end{enumerate}

Informiertheit (Lit.: Richter)\\
\\
Def.: h$_1$ und h$_2$ seien optimistisch, h$_2$ heißt besser (nicht schlechter) informiert als h$_1$, wenn für alle k $\in$K gilt
$$h_1(k)\leq h_2(k)$$
Folgerung: Jeder Knoten, der bei Verwendung von h$_2$ expandiert wird, wird auch bei Verwendung von h$_1$ expandiert.\\
Dann: K$^*$ - optimale Gesamtkosten\\
k wird expandiert bei $f_2(k)=g(k)+h_2(k)<K^*$\\
dann erst recht bei $f_1(k)=g(k)+h_1(k)<K^*$\\
\\
Verhalten des A$^*$-Algorithmus\\
Def.: h heißt monoton, wenn für alle Paare (k,n) $\in\mathbb{R}$\\
$$h(k)\leq e(k,n)+h(n)$$
Satz: Monotone Funktionen sind optimistisch\\
Bsp: einer opt. nicht monotonen Schätzfunktion
$$k\xrightarrow{4}n\xrightarrow{3}z$$
n(k)=6~~~~~~~~~~~~~~~~~~~h(n)=1\\
fast realistisch~~~~~~~~~gewachsener Optimismus\\
Satz: Es gibt für monotone h\\
\begin{enumerate}[(a)]
 \item $g(k)=g^*(k) fuer k\in CLOSED$
 \item k sei nach m expandiert worden $\rightarrow f(k)\leq f(m)$
 \item Wenn k expandiert wurde, gilt: $g^*(k)+h(k)\leq K^*$
 \item Jeder Knoten k mit $g^*(k)+h(k)<K^*$ wird expandiert
 \item Komplexität des A$^*$-Algorithmus'\\
	N - Anzahl der expandierten Knoten: Komplexität O(n)
\end{enumerate}
Hauptproblem: gute Funktion h\\
\\
3. Beispiel:\\
Von Franz holen
\\
\\
\subsubsection{A0$^*$-Algorithmus}

für AND-OR-Graphen\\
z.B. für Problemzerlegungen\\
Hyperkanten (Weglängen) zu bewerten\\
h muss Hyperkanten berücksichtigen\\
Lit.: Nilsson

\subsubsection{Übersicht zu den Suchverfahren}
\begin{tabular}{l|l}
Paradigmen			&	entspr. Verfahren\\\hline
Generieren und Testen		&	einfache uninformierte Suchverfahren\\
Differenzmethode		&	Verfahren mit Kostenfunktion (v.A. A$^*$)\\
Problemreduktion		&	A0$^*$-Alg.\\
\end{tabular}\\

Verfahren ohne Kostenfunktion
\begin{itemize}
 \item Tiefensuche
 \item Breitensuche
 \item Nichtdeterministische Suche
\end{itemize}

Verfahren mit Kostenfunktion\\
\\
\begin{tabular}{l|l}
Verfahrensname			&	Idee\\\hline
British-Museum-Prozedur		&	erschöpfende Suche\\
Hill climbing			&	Auswahl einer lokal besten Richtung\\
Strahlensuche			&	Verfolgen mehrerer Richtungen\\
Bestensuche			&	OPEN-Liste aller aktuellen nicht expandierten Knoten\\
Branch-and-Bound-Methode	&	Weglassen hoffnungsloser (Teil-)Wege\\
A-Alg.				&	Schätzfunktion h für den Restweg (von k aus)\\
A$^*$-Alg.			&	s.o.\\
A0$^*$				&	A$^*$-Erweiterung für AND-OR-GRAPHEN
\end{tabular}

\subsection{Strategische Spiele}
\subsubsection{Allgemeines zu strategischen Spielen}
Spiel wird als Graph dargestellt\\
\begin{description}
 \item[Knoten:] Spielzustände
 \item[Kanten:] Verbindung eines Spiel-Zustandes mit einem anderen, wenn Nachfolge-Zustand durch Aktion (zul. Zug) entsteht
 \item[Partie:] ein Weg durh den Baum (Schach: Weglänge $<$ 200)
\end{description}
$\rightarrow$ Menge der Partien im Graph vollst. repräsentiert -- bester Zug ablesbar

\subsubsection{Minimax-Strategie}
Idee: Auswahl des Zuges, der nach bestem Gegnerzug die beste Bewertung liefert\\
$\rightarrow$ folgende Aufgaben zu lösen:
$$z_j(z_i(s)) $$

\section{Klassische Logik-Ergänzungen - Elemente der klassischen Logik}
\subsection{Kalküle}
\begin{description}
 \item[Kalkül:] Syntax und Semantik einer Logik seien gegeben\\ ($\rightarrow$ Modell,  Folgerung)\\
		Hinzunahme einer Ableitungsrelation zum Nachbilden des Folgebegriffes
\end{description}

Korrektheit und Vollständigkeit des Kalküls von Interesse\\
\\
Korrektheit $\Leftrightarrow$ Was ableitbar ist (laut dem Kalkül), folgt auch\\
\\
Deduktion: Sprache 1. Ordnung gegeben\\
M-Menge aller log. Ausdrücke\\
W,W$_1$,W$_2$ - geg. Mengen logischer Ausdrücke\\
$\vdash$ -  Ableitungsfunktion\\
$A(W) = {a\in M | W \vdash a}$\\
$\vdash$ heißt deduktiv, wenn folg. Bedingungen erfüllt sind:
\begin{enumerate}
 \item Einbettung: $W\subseteq A(W)$
 \item Monotonie: $W_1\subseteq W_2 \Rightarrow A(W_1) \subseteq A(W_2)$
\end{enumerate}

Beispiele klassischer Deduktionsrelationen\\
\begin{enumerate}[(a)]
  \item Gentzen-Kalkül\\
	Anliegen: Formalisieren von im Ma gebräuchlichen Schlussregeln\\
	Einfachste Variante: 13 Regeln, kein Axiom\\
	W - Wissensbasis (Menge log. Ausdrücke)\\
	p(X), q(X) - log. Ausdruck, abh. von X\\
	$W \cup \{p\} \vdash q$, q ableitbar aus W unter Annahme p\\
	Schreibweise [p] q
	\begin{enumerate}[(1)]
	  \item	Und-Einführung\\
		$$\frac{p, q}{p\land q}$$ 
	  \item Und-Beseitigung
		$$\frac{p\land q}{p}, \frac{p\land q}{q}$$
	  \item Oder-Einführung
		$$\frac{p}{p\lor q}, \frac{q}{p\lor q}$$
	  \item Oder-Beseitigung
		$$\frac{p\lor q,[p]r,[q]r}{r}$$
	  \item Folgt-Einführung
		$$\frac{[p]q}{p\rightarrow q}$$
	  \item Folgt-Beseitigung (Modus Ponens, Abtrennungsregel)
		$$\frac{p,p\rightarrow q}{q}$$
	  \item	$\forall$-Einführung
		$$\frac{p(Y)}{\forall X p(X)}$$
		Bem.: p(Y) von keiner Anname abhängig, in der Y vorkommt
	  \item $\forall$-Beseitigung
		$$\frac{\forall X p(X)}{p(Y)}$$
	  \item $\exists$-Einführung
		$$\frac{p(Y)}{\exists x p(X)}$$
	  \item $\exists$-Beseitigung
		$$\frac{\exists X p(X),[p(Y)]q}{q}$$
	  \item $\neg$-Einführung
		$$\frac{[p]\Box}{\neg p}$$
	  \item	$\neg$-Beseitigung
		$$\frac{p, \neg q}{\Box}$$
	  \item Widerspruchsregel
		$$\frac{\Box}{p}$$
	\end{enumerate}
	\{1.,...,13.\} modelliert die intuitionistische Prädikatenlogik\\
	Kalkül korrekt, aber nicht vollständig\\
	Hinzunahme des (einzigen) log. Axioms $p\lor\neg p$\\
	\{1.,...,13.,$p\lor\neg p$\}
	Kalkül ist korrekt und vollständig
  \item	Klassischer Kalkül (Gödel)\\
	15 aussagenlogische Axiome (AXA)\\
	8 prädikatenlogische Schlussregeln (u.a. Modus Ponens)\\
	Korrektheit und Völlständig
  \item	Resolutionskalkül (SLD-Res. als Grundlage für Prolog)\\
	eine Schlussregel\\
	Korrektheit, Widerlegungsvollständigkeit (keine \glqq{}vollständige\grqq{} Vollständigkeit)
\end{enumerate}
Vergleich des deduktiven Schließens mit anderen Paradigmen\\
Urne mit Kugeln geg.\\
Syllogismen (Schlussfiguren)\\
\begin{description}
 \item[Deduktion:]	Alle Kugeln sind weiß.\\
			Diese Kugeln sind aus der Urne.\\
			Diese Kugeln sind weiß.
\end{description}
Schluss vom Allg. auf das Besondere\\
\begin{description}
 \item[Induktion:]	Diese Kugeln ins weiß.\\
			Diese Kugeln stammen aus der Urne.\\
			Dann sind alle Kugeln in der Urne weiß.\\
\end{description}
Schluss vom Besonderem auf das Allgemeine (kein korrektes Schließen)
\begin{description}
 \item[Abduktion:]	Alle K. in U. sind weiß\\
			Diese K. sind weiß\\
			Diese K. stammen aus U.
\end{description}
Umkehrung der Implikation, Suche nach Ursachen.
$$\frac{p\rightarrow q,q}{p}$$
$\rightarrow$ Schluss nicht korrekt\\
\\
Formalisierung des Gleichheitsbegriffs\\
Gleichheitsaxiome
\begin{description}
 \item[Reflexivität:] $\forall X\: X=X $
 \item[Symmetrie:] $\forall X\forall Y\: X=Y\rightarrow Y=X$
 \item[Transitivität:] $\forall X\forall Y\forall Z\: X=Y\land Y=Z\rightarrow X=Z$
\end{description}
Substitutionsaxiome für Funktions- und Prädiktensymbole
\begin{itemize}
 \item f - Fkt.-Symbol 
 \item p - Präd.-Symbol 
\end{itemize}
beide n-stellig\\
\\
$$\forall X_1 \dots \forall X, \forall Y \:\:X_i=Y\rightarrow f(x_1,\dots,x_i,\dots,x_n)=f(x_1,\dots,Y,\dots,x_n)$$
$$\forall X_1 \dots \forall X, \forall Y \:\:X_i=Y\rightarrow \{p(x_1,\dots,x_i,\dots,x_n)\rightarrow p(x_1,\dots,Y,\dots,x_n)\}$$

Probleme mit \glqq{}=\grqq{}:
\begin{itemize}
  \item Blindes Substituieren im Gl.-Beweisen $\Rightarrow$kombinatorische Explosion
  \item Resolution nicht umkehrbar
\end{itemize}
Auswege:
\begin{itemize}
 \item Paramodulation (Schlussregel)
 \item Termersetzungssysteme
\end{itemize}

\subsection{Herbrandt-Modelle}
Motivation: Zusammenhang zwischen Syntax und Semantik\\
Gegeben: Sprache 1. Ordnung (L), Basis B = [IS,PS,FS]
\begin{description}
 \item[IS:] Individuensymbole $\rightarrow$Konstantensymbole $\neq\emptyset$ oder Var.-Symbole
 \item[PS:] Prädikatensymbole
 \item[FS:] Funktionssymbole
 \item[Grundterm:] variablenfreier Term
 \item[Grundatom:] var.-freies Atom
 \item[Herbrand-Universum (HU):] Menge aller Grundterme von L
 \item[Herbrand-Basis (HB):] Menge aller Grundatome von L
\end{description}

Beispiel: f-Fkt.-Symbol - 3-stellig\\
Interpretation: f(I,X,Y) bezeichne das I-te Kind von den Personen X,Y\\
Grundterme: anna, franz, f(1,fritz,ina), f(7,f(3,paul,anna),max)\\
Grundatome: w(anna), oma(fritz,max), m(f(1,fritz,ina))\\
Herbrand-Interpretation: Interpretation, die
\begin{itemize}
 \item Symbol aus der Menge der Konstantensymbole eben dieses Symbol als Wert zuweist
 \item Symbol f $\in$ FS eine Abbildung aus HU$^n\rightarrow HU$ zuordnet, so dass $$(t_1,\dots,t_n)\in HU^n \rightarrow f(t_1,\dots,t_n)\in HU^n $$
 \item jedem Grundatom einen der Werte true oder false zuweist
\end{itemize}
P -  Menge abgeschlossener Ausdrücke (d.h. keine freien Var.)\\
Herbrand-Modell: H-Int., die Modell für P ist\\
Bem.: H\\



%KW 46 fiel ausgewählte

%KW 47
Herbrand-Interpretation HI1, HI2\\
\\
HI1: p(b), p(f(a)), p(f(c)), p(f(f(b))), p(f(f(f(a)))), $\dots$\\
q(a,b),q(a,c),$\dots$,q(c,c),q(a,f(a)),$\dots$\\
HI2: p(a), p(f(a),p(f(b)),p(f(c)),p(f(f(a))),$\dots$\\
q(f(a),a),$\dots$,q(f(f(b)),a),$\dots$\\
\\
Durchschnitt HI1 $\cap$ HI2 ist auch Modell für P
$$L=\{p(f(a)),p(f(c)),p(f(f(b))),\dots,q(a,f(a)),\dots\}$$
Satz 1: Der Durschnitt von H-Modellen ist H-Modell\\
\\
Satz 2: P-Menge von Programmklauseln, P habe ein Modell\\
$\rightarrow$ P hat ein H-Modell\\
Beweis: Int sei ein Modell für P. Dann ist
$$\text{Int} = \{p(t_1,\dots,t_n)\in HB: p(t_1,\dots,t_n)\}\text{ wahr bez. Int}$$
ein Modell für P.\\
\\
Satz 3: P wie oben. P ist genau dann unerfüllbar, wenn P kein H-Modell hat.\\
\\
Sätze 2 und 3 gelten nur im Falle von Prolog-Klauseln.\\
\\
%hier fehlt ein stück

\subsection{Gleichungswissen}
\subsubsection{Die Paramodulationsregel}
Geg.: 2 Klauseln (Disjunktion von Literalen)\\
$$\text{K}_1\text{:  }p_1\vee p_2\vee \dots \vee p_m$$
$$\text{K}_2\text{:  }r=s \vee q_2 \vee \dots \vee q_n$$
p$_1$ enthalte den Term t (möglicherweise als Teilterm), r und t unifizierbar, d.h.
$$\cup(r)=^{syn}\cup(t)$$
$\cup$ -- allgemeiner Unifikator\\
\\
p$_1$' ergebe sich aus p$_1$ durch Ersetzung von t durch s\\
\\
Paramodulant von K$_1$ und K$_2$:
$$\cup(p_1'\vee \dots \vee p_m \vee q_2 \vee \dots \vee q_n)$$
%hier fehlen Dinge
Satz: die Paramodulationsregel ist korrekt.

%hier fehlen ziemlich viele Dinge
%KW54(KW2)

\section{}

\glqq{}unless\grqq{} ist schon in Prolog: \glqq{}(backslash)+\grqq{}

\paragraph{TMS} (truth maintenance system)\\
zur Überprüfung bereits erfolgter Schlüsse (bei veränderten Annahmen)\\
Basis sind Protokolle über erfolgte Schlüsse\\
verschiedene Protokoll-Aufbereitungen möglich\\
TMS $\rightarrow$ JTMS (justification-based TMS) oder ATMS (assumption-based TMS)

\subsection{Einfache Revisionsmechanismen}
\begin{enumerate}
 \item Neuberechnung aller Schlussfolgerungen aus geänderten Ausgangsdaten (Zeit!)
 \item Speichern eines Protokolls von Schlussfolgerungen\\
	Neuberechnung von der Stelle im Protokoll aus, wo erste Änderung relevant wird
	\\$\rightarrow$höherer Speicheraufwand, gewissen Zeiteinsparung (unzureichend)
\end{enumerate}

\subsection{JTMS}

Protokoll-Aufbereitung bei Änderung von Annahmen\\
Nachvollziehen der Interferenzen\\
Streichen nicht mehr erlaubter Ableitungen\\
Abhängigkeitsnetz\\ %von Franz holen (Prüfungsrelevant)

\begin{tabular}{l|l|l}
Annahmen/Fakten			&	Regeln			&	Schlussfolgerungen	\\\hline
vs $\rightarrow$		&	(1) $\rightarrow$	&	vw			\\
				&	(2) $\rightarrow$	&	pb			\\
ab $\rightarrow$		&				&				\\
$\neg$ew $\rightarrow$		&	(3)$\rightarrow$	&				\\
\end{tabular}

Regel anwendbar $\Leftrightarrow$  alle positiven Rechtfertigungen ableitbar und alle neg. Hinderungsgründe nicht ableitbar.\\
Begründung-Verwendung: zeiteffektiver als Netz-Neuberechnung\\
Hauptproblem: Schleifen aller Art, dann Erweiterung des JTMS-Basisalg. erforderlich\\
\\
%Algorithmus siehe Franz
\\

\subsection{ATMS}

\section{Fuzzy Logik}
\subsection{Grundlegende Ideen und Begriffe}

Ausprägungsgrad einer Eigenschaft soll modellierbar sein.\\
\\
Bsp.: Lottogewinn(egon)\\
\\
Ausprägung dieser Eigenschaft wesentlich.\\
Möglichkeiten der Modellierung
\begin{itemize}
 \item Gewinn als 2. Ang. (falls bekannt)
 \item Zufallsgröße Gewinn
 \item Beschreibung von Lottogewinner als unscharfen Begriff (Fuzzy-Menge)
\end{itemize}

Aus.-Punkt: Aufweichung von \glqq{}$\in$\grqq{}\\
$x\in M$
\begin{itemize}
 \item gilt
 \item gilt ziemlich stark
 \item gilt ein wenig
\end{itemize}



Lit.: Zadek: Fuzzy Sets (1965)\\
\\
Fuzzy-Mengen-Begriff\\
\\
Geg. Grundmenge X\\
Funktion $\mu_M:\rightarrow[0,1]$\\
Die \glqq{}Fuzzy-Menge M in X\grqq{} wird gegeben durch $\{x,\mu_M(x)\}$\\
$\mu_M(x)$ heißt Zugehörigkeitsfunktion\\
Int.:
$\mu_M(x)$ bedeutet: $x\in M$ im Grade $\mu_M(x)$\\

Spezialfall der klassischen Menge:\\
$\uparrow$\\
$\mu_M(x)=1$ für $x\in M$;0 sonst\\
Statt $\{x,\mu_M(x)\}_{x\in X}$ wird M geschrieben\\
Bsp.: \glqq{}ein alter Mensch\grqq{}\\
\\
A,B geg. Fuzzy-Mengen\\
Fuzzy-Mengen-Relationen\\
\begin{description}
 \item[Teilmengen:] $A\subseteq B \leftrightarrow \forall_x\mu_A(x)\leq\mu_B(x)$
 \item[Gleicheit:] $A=B\leftrightarrow\forall_x\mu_A(x)=\mu_B(x)$
\end{description}
Operationen mit Fuzzy-Mengen:
\begin{description}
 \item[Komplement:] $B=\bar{A} \leftrightarrow \forall_x \mu_B(x) = 1-\mu_A(x)$
 \item[Vereinigung:] $\mu_{A\cup B}(x) = max\,\{\mu_A(x),\mu_B(x)\}$
 \item[Durchschnitt:] $\mu_{A\cap B}(x) = min\,\{\mu_A(x),\mu_B(x)\} $
\end{description}
Fuzzy-Optimierung:\\
Zielfunktion zu minimieren\\
Zulässiger Bereich ist jetzt unscharfe Menge\\
Optimum könnte knapp neben dem \glqq{}richtigen\grqq{} zulässigen Bereich liegen\\
Modellierung durch Fuzzy-Menge und höhere \glqq{}Kosten\grqq{} neben dem \glqq{}richtigen\grqq{} zulässigen Bereich\\
Z$\rightarrow Min $ neue Aufgabe\\
Z + Strafterm $\leftarrow$ umso größer, je mehr der \glqq{}richtige\grqq{} zul. Bereich verletzt wird.\\
\\
Fuzzy-Logik\\
Relationen bzw. Prädikate werden als Fuzzy-Mengen modelliert\\
$\rightarrow$ Graduierung  von Wahrheitswerten in [0,1]\\
Anwendung in der Steuerungstheorie\\
\\
Possibility-Theorie\\
eine Grundlage für das \glqq{}approximative Schließen\grqq{}\\
Regeln zur Komb. von Fuzzy-Regeln $\rightarrow$ mit Berechnung der Möglichkeit\\
\\
Satz 3.2






\end{document}