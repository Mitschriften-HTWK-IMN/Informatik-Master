\documentclass[a4paper,12pt]{article}
\usepackage[utf8]{inputenc}
\usepackage{amsmath}
\usepackage{amsfonts}
\usepackage{listings}
\usepackage{float}
\usepackage[ngerman]{babel}
\usepackage{url}


\title{Kryptologie - Geser}
\author{Hans-Christian Heinz\\
hheinz@imn.htwk-leipzig.de\\
CC-BY-NC-SA 4.0}
\date{WS 2012/13}

\pdfinfo{%
  /Title    (Kryptologie-Skript)
  /Author   (Hans-Christian Heinz)
  /Creator  ()
  /Producer (Prof. Alfons Geser)
  /Subject  ()
  /Keywords ()
}

\begin{document}
\maketitle

\tableofcontents

\section{Grundbegriffe}

Zwei Begriffe für Sicherheit im Englischen:
\begin{description}
 \item[Safety] Sicherheit vor Unfällen (\glqq{}technische Sicherheit\grqq{})
 \item[Security] Sicherheit vor Missbrauch (\glqq{}Zugriffssicherheit\grqq{})
\end{description}

Hier: Security.

Kryptographie (griech. crypto + graphein):=Geheimschrift \\
Zweck: Nachrichten übermitteln, wobei Vertraulichkeit und Integrität gewahrt bleiben.\\
\begin{itemize}
 \item Vertraulichkeit (engl. confidentiality) := kein Unbefugter kann die Nachricht verstehen
 \item Integrität (engl. integrity) := kein Unbefugter kann die Nachricht manipulieren
 \item Echtheit (engl. authenticity) := der Absender ist derjenige, der zu sein er behauptet
\end{itemize}

Integrität setzt Echtheit voraus: Ein falscher Absender kann eine beliebige Nachricht senden.\\
\\
Die Nachricht wird \underline{verschlüsselt}, d.h. in eine Form gebracht, die nur der Empfänger wieder verstehen kann. Der Empfänger kann die verschlüsselte Nachricht wieder \underline{entschlüsseln}, d.h. in lesbare Form übersetzen.\\
\\
Typische Anwender der Kryptographie:
\begin{itemize}
 \item Militär
 \item Geheimdienste (und Verschwörer)
 \item Industrie
 \item Liebende
\end{itemize}

Klartext (engl. plaintext):= zu übermittelnde Nachricht\\
Geheimtext (engl. ciphertext):= verschlüsselte Nachricht\\
\\
Der Klartext kann dabei eine beliebige Zeichenreihe sein.\\
\\
Gewünschte Eigenschaften einer Verschlüsselung:
\begin{itemize}
 \item Verschlüsselung und Entschlüsselung sind automatisch ausführbar
 \item Verschlüsselung und Entschlüsselung sind leicht änderbar
\end{itemize}

Deshalb implementiert man Verschlüsselung durch eine Verschlüsselungsfunktion mit \underline{Schlüssel} als Parameter. Ebenso implementiert man die Entschlüsselungsfunktion mit Schlüssel als Parameter.\
\\
Kryptosystem:=zueinander passende Ver- und Entschlüsselungsfunktion.\\
\\
Schreibweise:
\begin{itemize}
 \item $V_S$ Verschlüsselung mit Schlüssel S
 \item $E_S$ Entschlüsselung mit Schlüssel S
\end{itemize}
Schlüssel S für Verschlüsselung und Schlüssel S' für Entschlüsselung müssen zueinander passen, damit die Entschlüsselung gelingt:
$$E_{S'}(V_S(K))=K$$

Feste Namen für die \underline{Hauptfiguren} (engl. principals):
\begin{itemize}
 \item Alice = Absender
 \item Bob = Empfänger
 \item Charlie = Angreifer, Eindringling (engl. intruder)
\end{itemize}

Übertragungsmodell für verschlüsselte Nachrichten:\\
\\
\fbox{Alice}$\xrightarrow{K}$\fbox{V}$\xrightarrow{G}$\fbox{Kanal}\\
\\
%Grafik raussuchen
K:= Klartext\\
G:= Geheimtext\\
\\
Kerkhoffs Prinzip:= Die Algorithmen müssen öffentlich sein, nur die Schlüssel sind geheim.\\
Gründe:
\begin{itemize}
 \item Die Algorithmen sollen gut untersucht werden, um mögliche Schwächen offenzulegen. Viele Fachleute sollen die Algorithmen studieren und untereinander diskutieren (Forschung, Publikation)
 \item Algorithmen zu entwickeln, testen, installieren ist aufwändig und schwierig geheim zu halten (viele beteiligte Leute)
\end{itemize}

Kryptanalyse:= Geheimschrift-Analyse\\
Zweck: verschlüsselte Nachrichten ohne Kenntnis des Schlüssels verstehen\\
\glqq{} Schlüssel knacken\grqq{} \glqq{}Nachricht entziffern\grqq{}
Kryptologie:= Kryptographie und Kryptanalyse zusammengefasst.\\
\\
Bemerkung: Begriff Kryptologie drückt aus, dass Kryptographen die Methoden der Kryptoanalysten kennen müssen und umgekehrt.\\
\\
Arbeitsfaktor:= Aufwand zum Knacken des Schlüssels\\
\\
Angriff:= bestimmte Methode der Kryptanalyse\\
\\
Grundsätzliche Situation der Kryptoanalyse:
\begin{itemize}
 \item exhaustive Suche (:= alle möglichen Schlüssel probieren; engl. brute force) hat exponentiellen Aufwand in der Schlüssellänge.
 \item ausnutzbare Eigenschaften (Regelmäßigkeiten, Symmetrien) der Verschlüsselung können den Suchraum dramatisch verkleinern.
 \item Großer Suchraum ist notwendig, aber nicht hinreichend für Vertraulichkeit.
\end{itemize}

\paragraph{Ende KW41}

Mögliche Ausgangsituationen für einen Angriff:
\begin{description}
 \item[ciphertext-only:] Nur der Geheimtext liegt vor
 \item[known-plaintext:] Ein Paar bestehend aus Klartext und dezu gehörigem Geheimtext liegt vor
 \item[chosen-plaintext:] Der Angreifer kann sich einen Klartext aussuchen und bekommt den Geheimtext dazu
\end{description}

\underline{Bem.:} Nur vertrauliche Nachrichten verschlüsseln. Insbesondere keine Nachrichten verschlüsseln, die sich Charlie aus anderen Quellen leicht rekonstruieren kann, z.B. einen Wetterbericht. Solche Nachrichten liefern ihm known-plaintext.\\
Mögliche Eingriffe des Angreifers:
\begin{enumerate}
 \item Geheimtext lesen und speichern
 \item den Empfang des Geheimtextes stören
 \item Gespeicherte Nachricht nochmal schicken (engl.: Replay)
 \item Fingierte Geheimtexte einfügen
 \item Geheimtext abfangen und geändert weitergeben
 \item Sich als Bob ausgeben
\end{enumerate}


\section{Symmetrische Kryptosysteme}

\underline{Symmetrisches Kryptosystem:} Zueinander passende Schlüssel sind gleich.\\
Verschlüsselung und Entschlüsselung sind durch denselben Schlüssel codiert. Alice und Bob müssen den Schlüssel kennen (\glqq gemeinsamer Schlüssel\grqq).

\subsection{Substitution}
\underline{Chiffre:} Verschlüsselungsverfahren, das zeichenweise arbeitet\\
\underline{Monoalphabetische Substitution:} Ersetzen von Zeichen durch andere Zeichen nach einer gegebenen Abbildungsvorschrift.\\
Abbildungsvorschrift typisch durch Tabelle gegeben.\\
Vorsicht: Hier Alphabet = Tabelle!\\
Beispiel: Caesars Chiffre.\\
Jeden Buchstaben durch den Buchstaben um 3 Stellen weiter im Alphabet ersetzen, also z.B. A durch D, B durch E, ... W durch Z, X durch A, Y durch B, Z durch C.\\
\begin{table}[h]
\centering
\begin{tabular}{l|c c c c c c}
Zeichen im Klartext & A & ... & W & X & Y & Z\\ \hline
Zeichen im Geheimtext & D & ... & Z & A & B & C\\
\end{tabular}
\end{table}
 Klartext: MORGENANGRIFF\\
 Geheimtext: PRUJHQDJULII\\
 \\
 Entschlüsselung: Umkehrabbildung (:= Tabelle von unten nach oben lesen)\\
 \\
Allgemeine: beliebige Permutation des Alphabets\\
Zur Erinnerung:\\
Permutation = umkehrbare Abb. einer endl. Menge auf sich selber. Es gibt n! (\glqq n-Fakultät\grqq) verschiedene Permutationen auf einer Menge mit n Elementen\\
Beispiel: Der ASCII-Zeichensatz (128 Zeichen) erlaub 128! Permutationen.

Trotzdem sind Substitutionen einfach zu knacken, nämlich durch Häufigkeitsanalyse. Die Sprache des Klartexts muss dazu bekannt sein.\\
\underline{Häufigkeitsanalyse} im Deutschen:
\begin{enumerate}
 \item In einem deutschen Text kommen Leerzeichen und Buchstabe e durchschnittliche mit einer Häufigkeit von je 17\% vor; danach kommt n mit 10\% vor.
 \item Wenn der Klartext ausreichend lang ist, darf man dort etwa dieselben Häufigkeiten erwarten (\glqq Gesetz der großen Zahlen\grqq).
 \item Charlie stellt ein Histogramm auf (:=Tabelle der Häufigkeiten). Die häufigsten Zeichen im Geheimtext sind wahrscheinlich die Verschlüsselung con Leerzeichen und e.
\end{enumerate}
Verfeinerung: Analyse der Häufigkeit von Paaren (\glqq{}Bigrammen\grqq) und Tripel (\glqq Trigrammen\grqq) von Buchstaben. Im Deutschen: er, en, ch, de, sch, der, ich, ein, che.\\
Die Häufigkeitsanalyse profitiert von der Redundanz im Klartext. Die Redundanz kann praktisch eliminiert werden durch Kompression des Klartexts (z.B. mit dem Ziv/Lempel-Algorithmus: zip, gzip).

\subsection{Transposition}
\underline{Transposition:} Ersetzen der Position eines Zeichens durch eine andere Position nach einer gegebenen Abbildungsvorschrift.\\
Die Positionsmenge ist endlich und liegt fest. Deshalb kann man nur Klartext fester Länge so verschlüsseln.\\
Der Klartext wird dazu in Blöcke der vorgegebenen Länge L zerlegt. Die blockweise Verschlüsselung nennt man auch \underline{Blockchiffre}.\\
Beispiel:\\
Schlüsselbegriff: \glqq{}Basis\grqq
\begin{table}[h]
\centering
\begin{tabular}{l|c c c c c c}
Position im Klartext & 1 & 2 & 3 & 4 & 5 \\ \hline
Position im Geheimtext & 2 & 1 & 4 & 3 & 5\\
\end{tabular}
\end{table}

Klartext: MORGE'N ANG'RIFF '\\
Geheimtext: OMGRE' NNAG'IRFF '\\
\\
Zusätzliche Maßnahme: Die Blöcke zeilenweise untereinander hinschreiben und spaltenweise auslesen (\glqq{}columnar transposistion\grqq).\\
\\
Hier:
\begin{table}[h]
\centering
\begin{tabular}{l l l l l}
O & M & G & R & E\\
 & N & N & A & G\\
I & R & F & F & 
\end{tabular}
\end{table}

\subsection{Lineare Transformation}
Claude Shannon führte folgende Begriffe ein:\\
\begin{description}
 \item[Confusion (deutsch: Verwirrung):] zwischen Geheimtext und Schlüssel muss ein möglichst komplizierter Zusammenhang bestehen.
 \item[Diffusion (deutsch: Streuung/Durchdringung):] Das Umklappen eines Bits im Klartext (d.h. die Änderung von 0 nach 1 oder umgekehrt) muss das Umklappen jedes Bits im Geheimtext mit einer Wahrscheinlichkeit von 0.5 bewirken.
\end{description}

Confusion soll verhindern, dass Charlie aus dem Geheimtext Informtation über den Schlüssel bezieht.\\
\\
Intuitiv:\\
Diffusion streut die Wirkung eines Bits soweit wie möglich. Die Verringerung des Arbeitsfaktors soll verhindert werden.\\
Die monoalphabetische Substitution hat keine Diffusion: Das Umklappen eines Klartext-Bits wirkt sich nur auf die Bits eines Zeichens im Geheimtext aus.\\
\\
Zur Erinnerung:\\
Eine Transformation f:V$\rightarrow$V auf einem Vektorraum(V,K,+,*) heißt \underline{linear}, wenn sie die Linearitätseigenschaften hat:\\
$$f(a+b)=f(a)+f(b)$$ für alle a,b $\in$ V
$$f(\alpha *a) = \alpha *f(a)$$ für alle a$\in$ V, $\alpha\in$ K \\
\\
Wenn eine Basis des Vektorraums gegeben ist, kann ein Vektor als Koordinatentupel dargestellt werden, und eine lineare Transformation kann ausgedrückt werden durch Multiplikation des Vektors mit einer Matrix.\\
Ein Block der Länge L kann als Vektor der Dimension L aufgefasst werden. Die Zeichen sind dabei die Skalare. Verschlüsselung:
$$G = A*K$$
wobei A eine umkehrbare L$\times$L-Matrix ist, d.h. det(A)$\neq0$. Entschlüsselung:
$$K = A^{-1}*G $$
Die Matrix A ist hier der Schlüssel.\\
\\
Hill-Chiffre =\\
Lineare Transformation im Vektorraum $\mathbb{Z}_{26}^L$.\\
Geeignet entworfende lineare Transformationen haben Diffusion.\\
Angriffsmöglichkeit:
$L^2$ known-plaintext-Paare, d.h. L Paare von Blöcken reichen aus für ein lineares Gleichungssystem, dessen Unbekannte die $L^2$ Koeffizienten der Matrix sind.\\
Abhilfe: Kombination der linearen Transformation mit nicht-linearen Verschlüsselungsmethoden, z.B. mit einer geeigneten Substitution.
\underline{Bem.:} Transpositionen helfen nicht, denn jede Transposition ist als lineare Transformation darstellbar.

\subsubsection{Bitweises Exklusiv-Oder}

Der Klartext sei als endliche Bitfolge gegeben. Der Schlüssel muss dazu eine Bitfolge mit derselben Länge sein. Der Schlüssel wird bitweise mit dem Klartext Exklusiv-Oder (XOR) verknüpft.\\
Beispiel:
\begin{table}[h]
\centering
\begin{tabular}{l l l l l l l}
1 & 0 & 1 & 1 & 1 & 0 & Klartext\\
0 & 1 & 1 & 0 & 1 & 1 & Schlüssel\\\hline
1 & 1 & 0 & 1 & 0 & 1 & Geheimtext
\end{tabular}
\end{table}

Entschlüsselung: Geheimtext mit Schlüssel bitweise EXOR verknüpfen.\\
Schlüsselstrom (engl.: keystream):= Schlüssel als unendliche Bitfolge.\\
One-time-pad := zufällig generierter Einweg-Schlüssel\\
Wenn der Schlüssel zufällig ist, d.h. seine Bits  zufallsverteilt sind, kann die Veschlüsselung nicht geknackt werden.
Problem: die benötigte Schlüssellänge ist groß Der Schlüssel kann erbeutet werden.
\\
%KW45 fehlt, siehe Mail von Geser
\\
Die Durchführung dieser Ausschreibung gilt als vorbildlich. Gewinner war der Algorithmus Rijndael (nach den belgischen Autoren Vincent Rijmen und Joan Daemen).\\
Offizielle AES-Spezifikation:\\
\url{http://csrc.nist.gov/publications/fips/fips197/fips-197.pdf}\\
Algorithmus von AES-128 in Pseudocode:
\begin{enumerate}
  \item Parameter:\\
	Klartext K: Feld der Länge 16 von Bytes (16 Bytes = 128 Bits)\\
	Schlüssel S: Feld der Länge 16 von Bytes
  \item Variablen:\\
	Textregister T: 2-dimensionales Feld von Bytes mit 4 Zeilen und 4 Spalten
  \item Kopiere K spaltenweise in T
  \item Verknüpfe T bitweise EXOR mit Rundenschlüssel S$_0$.
  \item Wiederhole für Runde $r\in\{1,\dots,10\}$ folgendes:
  \item (Anfang der Schleife) SubBytes: Substituiere T mit vorgegebener Substitution auf Bytes
  \item Rotate Rows: Für jede Zeile i, rotiere die i-te Zeile zyklisch nach links, um i Stellen
  \item MixColumns: Mische T, d.h. multipliziere jede Spalte mit einer vorgegebenen 4x4-Matrix, falls r$\neq$10
  \item AddRoundKey: Verknüpfe T bitweise EXOR mit dem Rundenschlüssel S$_r$ (Ende der Schleife)
  \item Ergebnis: Lese Inhalt von T spaltenweise aus
\end{enumerate}

Der Rundenschlüssel S$_r$, für $r\in\{0,1,\dots,10\}$ ist ein 2-dim. Feld von Bytes mit 4 Zeilen und 4 Spalten. Er wird aus S und r berechnet.\\
Zur Effizienz:\\
\glqq{}Gute Software-Implementierungen auf einem 2GHz-Prozessor erreichen 700MBit/s Verschlüsselungsrate. Das reicht um 100 MPEG-2 Videos in Echtzeit zu verschlüsseln\grqq{}(nach Tanenbaum)\\
\\
Mathematischer Hintergrund:
\begin{enumerate}
  \item	$\mathbb{Z}_2$: Restklassenring der ganzen Zahlen modulo 2.\\
	$\mathbb{Z}_2 = \{0,1\}$. Addition = Subtraktion = EXOR.\\
	Multiplikation = UND. $\mathbb{Z}_2$ Körper, $1^{-1}=1.$\\
	Darstellung von 0,1 durch Bits.
  \item	K Ring der Polynome über $\mathbb{Z}_2$ modulo $x^8+x^4+x^3+x+1$.\\
	Also K = $\mathbb{Z}_2[x]/(x^8+x^4+x^3+x+1)$. Das Polynom $x^8+x^4+x^3+x+1$ ist irreduzibel, also ist K ein Körper.\\
	Die Elemente von K sind Polynome höchstens von Grad 7.\\
	Darstellung als Bytes in Bitnotation oder Hexadezimal-notation.
  \item P Ring der Polynome über K modulo $x^4$+1.\\
	Das Polynom $x^4+1$ ist reduzibel, also ist P kein Körper.\\
	Die Elemente von P sind Polynome höchstens von Grad 3.\\
	Darstellung als Maschinenwörter (32 Bit) oder  Vierergruppe von Paaren von Hexadezimalziffern.\\
	Manche Elemente haben Inverse:
	$$03\:01\:01\:02\text{ hat Inverses }0B\:0D\:09\:0E$$
\end{enumerate}
	Multiplikation in P mit festem Polynom ($a_3,a_2,a_1,a_0$) berechnet weden durch Multiplikation einer festen Matrix mit dem Vektor.\\
	Sei:
	$$(d_3,d_2,d_1,d_0)=(a_3,a_2,a_1,a_0)\odot(b_3,b_2,b_1,b_0)$$
	Dann gilt:
	$$
	  \begin{pmatrix}
	    d_{0}\\
	    d_{1}\\
	    d_{2}\\
	    d_{3}
	  \end{pmatrix}
	  =
	  \begin{pmatrix}
	    a_{0} & a_{3} & a_2 & a_{1} \\
	    a_{1} & a_{0} & a_3 & a_{2} \\
	    a_{2}  & a_1  & a_0 & a_3  \\
	    a_{3} & a_{2} & a_1 & a_{0}
	  \end{pmatrix}
	  \odot
	  \begin{pmatrix}
	    d_{0}\\
	    d_{1}\\
	    d_{2}\\
	    d_{3}
	  \end{pmatrix}
	  $$
Matrix für MixedColumns:
$$
	  \begin{pmatrix}
	    02 & 03 & 01 & 01 \\
	    01 & 02 & 03 & 01 \\
	    01 & 01 & 02 & 03 \\
	    03 & 01 & 01 & 02
	  \end{pmatrix}
$$
Einzelheiten zu Subbytes:
$\sigma: K\rightarrow K$ sei def. durch 
$$  \sigma_1(n) = \left\{ 
  \begin{array}{l l}
    0 & \quad \text{falls $n=0$ }\\
    n^{-1} & \quad \text{sonst - hier steckt die Körper-Eigenscchaft von K}\\
  \end{array} \right.
$$
$\sigma_2:K\rightarrow K $ ist def. durch:
$$\sigma_2(n)=1F_{16}\odot n\oplus63_{16}  $$
$$\sigma_2(n)=(1F*k)mod101_{16}\oplus63_{16} $$
(Affine Abbildung auf K)\\
$\sigma:K\rightarrow K$ ist def. durch:
$$\sigma(n) = \sigma_2(\sigma_1(n))$$
$\sigma_1$ liefert nicht-linearen Anteil\\
$\sigma_2$ sorgt für Diffusion innerhalb eines Bytes.\\
z.B. $\sigma(00_{16} = 63_{16})$.\\
\\
Berechnung der Rundenschlüssel (engl. key expansion, key schedule):\\
S$_{r,i,j}$: Element in Zeile i, Spalte j des Rundenschlüssels S$_r$
\begin{enumerate}
 \item S$_0$ wird mit S spaltenweise aufgefüllt.
 \item Für jede Runde $r\in\{1,\dots,10\}$ wiederhole:
 \item Für jede Zeile $j\in\{0,1,2,3\}$ wiederhole:
 \item $S_{r,i,0}:=S_{r-1,i,0}\oplus\sigma(S_{r-1,(i+1)mod4,3})\oplus C_{r,i})$\\
	wobei
      $$C_{r,i}:= \left\{ 
	  \begin{array}{l l}
	  x^{r-1} & \quad \text{falls $i=0$ }\\
	  0 & \quad \text{sonst}\\
	\end{array} \right.
      $$
  \item Für jede Spalte $j\in\{1,2,3\}$ wiederhole:
  \item Für jede Zeile $i\in\{0,1,2,3\}$ wiederhole:
  \item $S_{r,i,j}:=S_{r-1,i,j}\oplus S_{r,i,j-1}$
\end{enumerate}


\section{Schlüsselaustausch und Chiffre-Moden}
\subsection{Diffie/Hellman-Schlüsselaustausch}
Autoren: Whitfield Diffie, Martin Hellman, Ralph Merkle (1976)\\
Ablauf:
\begin{enumerate}
 \item Alice und Bob einigen sich öffentlich auf zwei große Zahlen n und g. Dabei sei n eine Primzahl und $\frac{n-1}{2}$ eine Primzahl. ($\frac{n-1}{2}$ ist eine \underline{Sophie-Germain-Primzahl}\\
 g sei eine \underline{Primitivwurzel}, d.h. alle Zahlen zwischen 1 und n-1 lassen sich modulo n als Potenzen von g darstellen.\\
 z.B. Alice wählt solche n und g und sendet sie unverschlüsselt an Bob.
 \item Alice wählt eine große Zahl x und hält sie geheim.\\
	Bob wählt eine große Zahl y und hält sie geheim.
 \item Alice berechnet $g^x$ mod n und sendet das Ergebnis unverschlüsselt an Bob.
 Bob berechnet $g^y$ mod n und sendet das Ergebnis unverschlüsselt an Alice.
 \item Alice berechnet ($g^y$ mod n)$^x$ mod n = $g^{xy}$ mod n
 Bob berechnet ($g^x$ mod n)$^y$ mod n = $g^{xy}$ mod n
\end{enumerate}

$g^{xy}$ ist der gemeinsame Schlüssel von Alice und Bob.\\
Charlie kenn n,$g$,$g^x$ mod n,$g^y$ mod n. Aber er kennt nicht x, y, $g^{xy}$ mod n. Um x aus $g^x$ mod n zu berechnen, muss Charlie den Logarithmus modulo n zur Basis g auf $g^x$ mod n anwenden

\subsection{Stromchiffren}

Schlüsselstromgenerator := Algorithmus, der aus einem vorgegebenen Schlüssel einen Schlüsselstrom erzeugt.\\
\\
Schlüsselstrom kann verwendet werden für bitweises EXOR: \glqq{}Stromchiffre\grqq{}.\\
Typische Architektur:\\
Zustandsautomat ohne Eingabe mit Ausgabe. Zustand = Bitvektor mit vorgegebener Länge n.\\
Vorgegebene Bestandteile:
\begin{description}
 \item[init:] Zustand
 \item[f:] Zustand$\rightarrow$Zustand
 \item[g:] Zustand$\rightarrow\mathbb{B}$
\end{description}
Pseudocode für einen Schlüsselstromgenerator:
\begin{enumerate}
 \item z:=init
 \item Endlos wiederholen:
 \item z:=f(z)
 \item Ausgabe g(z)
\end{enumerate}

Es gibt $2^n$ verschiedene Zustände. Nach spätestens $2^n$ Schritten wiederholt sich ein Zustand. Periodisches Verhalten mit Periode $\leq2^n$.\\
\\
Forderungen an einen Schlüsselstromgenerator:
\begin{enumerate}
 \item f und g sind Einweg-Funktionen. d.h. f(z) leicht berechenbar aus z. Urbild z von f(z) praktisch nicht berechenbar.
 \item Die Periode ist viel größer als das verwendete Segment (Faustformel: mindestens um den Faktor 100).
\end{enumerate}

\subsection{Chiffre-Moden}

Angriffsmöglichkeiten bei Blockchiffren:\\
Gleicher Klartextblock liefert gleichen Geheimtextblock.\\
Beispiel: Weihnachtsgeld
\begin{description}
 \item[$k_1$:] Boss
 \item[$k_2$:] 10.000 Euro
 \item[$k_3$:] Alice
 \item[$k_4$:] 500 Euro
 \item[$k_5$:] Charlie
 \item[$k_6$:] 5 Euro
\end{description}
Charlie kann $G_6$ durch z.B. $G_4$ ersetzen.\\
\\
Abhilfe: Chiffre-Modus:= Klartextblöcke nicht immer gleich verschlüsseln.\\
\begin{enumerate}
 \item Naive Lösung: Electronic Code Book Mode (ECB)
 \item Cipher BLock Chaining (CBC)\\
 Statt $K_i$ wird $K'_i$ verschlüsselt, wobei
 $$  K'_1 := \left\{ 
  \begin{array}{l l}
    IV\oplus K_0 & \quad \text{,falls $i=0$ }\\
    G_{i-1}\oplus K_i & \quad \text{sonst}\\
  \end{array} \right.
$$
Initialisierungsvektor(IV):= Parameter, der Verschlüsselung variiert.\\
IV wird willkürlich gewählt und im Klartext mit übertragen. Die Variierung soll die Angriffsfläche verringern.
%Abbildung Verschlüsselung mit CBC-Modus
\item Cipher Feedback Mode (CFB):\\
Geeignet für Dialoge, byteweise Kommunikation. Um das i-te Byte zu verschlüsseln, wird ein 128-Bit-Block aus den Bytes $i-16,\dots,i-1$ des Geheimtexts (der IV liefert die Geheimtextbytes mit Nummer $-16,\dots,-1$) gebildet, mit der Blockchiffre verschlüsselt, und das niedrigstwertige Byte des Ergebnisses mit dem Klartextbyte bitweise EXOR-verknüpft.\\
Beispiel: i=42. Das 42. Klartextbyte soll verschlüsselt werden.\\
Geheimtextblock: $g_{26},\dots,g_{41}$\\
Ergebnisse der Verschlüsselung mit der Blockchiffre: $g'_{15},\dots,g'_{0}$\\
$K_{42}\oplus g'_0$ ist das Ergebnis.
\end{enumerate}



\section{Asymmetrische Kryptosysteme}

\section{Digitale Unterschriften}

\section{Streuwerte}

\section{Schlüsselverwaltung}

\section{Beglaubigungprotokolle}

\section{Sichere Kommunikation}


\begin{thebibliography}{9}
\bibitem{Tanenbaum}
Andrew S. Tanenbaum: Computer Networks, Pearson 2003

\bibitem{Handbook}
A.J. Menezes, P.C. van Oorschot, S.A. Vanstone: Handbook of Applied Cryptography, CRC Press 2002
\end{thebibliography}


\end{document}
